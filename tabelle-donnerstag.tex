\newpage
\renewcommand{\arraystretch}{1.4}
\renewcommand{\conferenceDay}{\donnerstag}
\section*{Vorträge am Donnerstag}\label{donnerstag}
\enlargethispage{1\baselineskip}
\setPageBackground
  \noindent\begin{tabular}{lZ{2.0cm}Z{2.0cm}Z{2.0cm}}
    & \multicolumn{1}{c}{\cellcolor{geoblau} Rundbau}
    & \multicolumn{1}{c}{\cellcolor{hellgelb} Anatomie}
    & \multicolumn{1}{c}{\cellcolor{hellgruen} Weismannhaus}
      \tabularnewline
      \rowcolor{commongray}
      07:00
      & \multicolumn{3}{c}{Frühsport (siehe Seite~\pageref{fruehsport})}
      \tabularnewline
      09:00
      \talk{Vorbereitung einer Großveranstaltung mit QGIS und QField}{Jelto Buurman, Jan Klein}
      \longTalk{2}{\emph{Demo-Session (60 Min.)}\linebreak
        Visualisierung und Analyse von Satellitenbildern mit der EnMAP-Box}{Andreas Rabe}
      \talk{OSM-Daten verarbeiten mit Python und Pyosmium}{Sarah Hoffmann}
      \tabularnewline
      09:30
      \talk{Aufbereitung von Vektordaten als Grundlage für Landnutzungsmodelle}{Mirko Blinn}
      &
      \talk{OSM-Daten mit Vektortiles erfolgreich nutzen}{Robert Klemm}
      \tabularnewline
      10:00
      \talk{Vektorverschneidung mit QGIS}{Marco Hugentobler}
      \talk{Offene Smart Farming Produkte aus offenen Satellitendaten}{Sven Bingert et al.}
      \talk{OSMPOIDB, eine kontinuierlich aktualisierte POI"=Datenbank auf OSM-Basis}{Sven Geggus}
      \tabularnewline
      \rowcolor{commongray}
      10:30 & \multicolumn{3}{c}{%
        \parbox[c]{24pt}{%
          \includegraphics[height=10pt]{cafe}%
        }
      Kaffeepause}\tabularnewline
    \end{tabular}

  \newpage
  \noindent\begin{tabular}{lZ{2.0cm}Z{2.0cm}Z{2.0cm}}
    & \multicolumn{1}{c}{\cellcolor{geoblau} Rundbau}
    & \multicolumn{1}{c}{\cellcolor{hellgelb} Anatomie}
    & \multicolumn{1}{c}{\cellcolor{hellgruen} Weismannhaus}
      \tabularnewline
      11:00
      \talk{Einsatz von XPlanung in der kommunalen Praxis~-- ein Werkstattbericht}{Torsten Friebe, Michael Schulz}
      \talk{Neuentwicklung der GDI-DE-Testsuite}{Marc Jansen, Manuel Fischer}
      \talk{Gefahrenbewertung im Radverkehr mittels Crowdsourcing~\noVideo}{Rafael Hologa, Nils Riach}
      \tabularnewline
      11:30
      \talk{\cellcolor{change}\emph{Demosession\linebreak(30 Min.)}\linebreak%
      Verarbeitung von Gelände\-daten mit QGIs}{Jelto Buurman}
      \talk{Der neue OGC-API-Standard ist da!}{Pirmin Kalberer}
      \talk{Entwicklung des Berliner Radverkehrs anhand von öffentlich gemachten Verkehrszähldaten~\noVideo}{Simon Metzler}
      \tabularnewline
      12:00
      \talk{Map-Editor für individuelle amtliche Vektorkarten}{Sebastian Ratjens}
      \talk{MapServer-Statusbericht}{Jörg Thomsen}
      \talk{Lightning Talks}{}
      \coffeespace\tabularnewline
      \rowcolor{commongray}
      12:30 & \multicolumn{3}{c}{%
        \parbox[c]{24pt}{%
          \includegraphics[height=10pt]{restaurant}%
        }
      Mittagspause}
      \tabularnewline
    \end{tabular}

  \newpage
  \noindent\begin{tabular}{lZ{2.0cm}Z{2.0cm}Z{2.0cm}}
    & \multicolumn{1}{c}{\cellcolor{geoblau} Rundbau}
    & \multicolumn{1}{c}{\cellcolor{hellgelb} Anatomie}
    & \multicolumn{1}{c}{\cellcolor{hellgruen} Weismannhaus}
      \tabularnewline
      13:30
      \talk{QGIS"=Kartografie"=Verbesserungen 2019}{A. \mbox{Neumann}}
      \talk{Schneller, besser, leichter~-- PostGIS~3}{Felix Kunde}
      \talk{Geoprocessing mit OpenCaching}{Matthias Hinz}
      \tabularnewline
      14:00
      \talk{TEAM Engine: Vorstellung der neusten Tests für OGC-Standards}{Dirk Stenger}
      \talk{Verbindungen schaffen mit PostgreSQL"=Foreign"=Data"=Wrappern}{Astrid Emde}
      \talk{Räumliche Verortung von textbasierten Social"=Media"=Einträgen}{Svenja Ruthmann et al.}
      \tabularnewline
      14:30
      \talk{OSGeo-Projekt dee\-gree 2020~-- Neuigkeiten}{Torsten Friebe, Dirk Stenger}
      \talk{AD und PostgreSQL"=Rollen verknüpfen mit dem Höllenhund}{Michael Schulz}
      \talk{Lightning Talks}{}
      \tabularnewline
      \rowcolor{commongray}
      15:00 & \multicolumn{3}{c}{%
        \parbox[c]{14pt}{%
          \includegraphics[height=10pt]{photo}%
        }
        Fototermin (Eingang Mensa)
        \hspace{1em}
        \parbox[c]{14pt}{%
          \includegraphics[height=10pt]{cafe}%
        }
      Kaffeepause}
      \tabularnewline
      15:30
      \talk{OpenLayers: v6.x und wie es weitergeht}{Marc Jansen, Andreas \mbox{Hocevar}}
      \talk{FOSS in der Cloud}{Daniel Koch, Carmen \mbox{Tawalika}}
      \talk{Routenplanung mit BRouter und BRouter-Web}{A. Brenschede, N. Renner}
      \tabularnewline
    \end{tabular}

  \newpage
  \noindent\begin{tabular}{lZ{2.0cm}Z{2.0cm}Z{2.0cm}}
    & \multicolumn{1}{c}{\cellcolor{geoblau} Rundbau}
    & \multicolumn{1}{c}{\cellcolor{hellgelb} Anatomie}
    & \multicolumn{1}{c}{\cellcolor{hellgruen} Weismannhaus}
      \tabularnewline
      16:00
      \talk{Wegue~-- OpenLayers und Vue.js in der Praxis}{Jakob Miksch, Christian Mayer}
      \talk{GRASS GIS in der Cloud: Actinia-Geo\-prozessierung}{Markus Neteler, Carmen \mbox{Tawalika}}
      \talk{PTNA: Qualitätssicherung für ÖPNV in OpenStreetMap}{Toni Erdmann}
      \tabularnewline
      16:30
      \talk{Javascript-Bibliotheken zur Einbindung von historischen Umwelt- und Klimainformationen}{Michael Kahle}
      \longTalk{2}{\emph{Demo-Session (60 Min.)}\linebreak
      Das OSGeo-Datacube-Community-Projekt Rasdaman}{pebau}
      \talk{Automatische Korrektur von ÖV-Stationen in OSM}{Patrick Brosi}
      \tabularnewline
      17:00
      \talk{Neues vom GeoStyler}{Christian Mayer, Jan Suleiman}
      &
      \talk{Qualitätsbewertung von OSM"=Gebäudedaten}{Leonie Möske}
      \tabularnewline
      \rowcolor{commongray}
      18:00
      &
      &
      \multicolumn{1}{Z{2.00cm}}{\cellcolor{dezentrot}FOSSGIS-Mitglieder\-versammlung \noVideo}
      &
    \end{tabular}
\renewcommand{\arraystretch}{1.0}
\justifying
\setPageBackground
